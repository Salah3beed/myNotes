\documentclass[11pt]{article}
\usepackage[utf8]{inputenc}
\usepackage[margin=0.5in]{geometry}
\begin{document}
\setcounter{section}{0}
\begin{center}
{\LARGE \textbf{German University In Cairo
	\\ SalahEl-Din Ebeed 
	\\ 48-- Mechatronics
}
\\}  find me anywhere (literally) @salah3beed
\end{center}
\section{Research Body}
\begin{itemize}

\item Abstract: Relevance - Purpose - Participants - Experiment 
\item Introduction: Lit. Rev - Purpose(Aim) - Gap - Operational Defs. Research Questions\&/Hypothesis. 
\item Methodology: Sample - Instruments - Procedures 
\item Results: Results in relation to  RQ - Significance - p value.
\item Discussion/Conclusion : Interpretation of results(Verbal) - Limitations - Recommendations for future research - Practical implications.
\end{itemize}

\section{Article Review}
\subsection{Citation}

\subsection{Research Design}
\begin{itemize}

	\item Basic (theoretical) or Applied (experimental)
	\item Exploratory (has a research question) or Confirmatory (has a hypothesis)
	\item Quantitative (statistics) or Qualitative (no statistics)

\end{itemize}
\subsection{Operational Defections}
Defections of the independent  and dependant variables.
\subsection{Gap/Significance}
Gap = what is messing in the previous researches.\\
if there's no Gap, search for the importance of the investigation, this can be found in the problem that led to the investigation or the conflicts shown in previous research.
\subsection{Aim}
The purpose of the study ? Why it is made ?
it should include the dependant and independent variables.\\
Can be found in the \textbf{Abstract/Introduction} 
\subsection{Sample}
Size, Characteristics, Rationale of selection (random or convenient) 
\subsection{Research Questions/Hypothesis}
Look for them, if not found infer your own research questions in "wh" form, it should included the ind/dep variables. 
\subsection{Instruments}
The data gathering procedures; surveys, tests, observations. They must have \textbf{Validity} (accurate) and \textbf{Reliability} (consistent results)
\\
Name, Purpose(what does it measure?), Description (number of items, scale, subcomponents,.....)
Example (Name: Survey, Description: it consisted of 5 parts each part had 2 questions, the first question was in scale from 1 to 5 and soooooo on. )

\subsection{Procedures}
How the study was conducted, the story from the beginning to the end in terms of steps.\\
Type: \begin{itemize}

\item Cross-sectional: data was gathered at a specific point in time.
\item Longitudinal: repeated observations of the same variables over short/long amounts of time.
\item Experimental: where researchers introduce an intervention and study the effects. VS Observational (researchers serve the effect of an intervention without trying to change who is or is not exposed to it.
\end{itemize}
It contains the steps, approvals taken, how was each instrument used, and what information was told to the sample, how were they have treated ? were they anonymous ?
\\ Triangulation of methods: using more than two instruments to measure the variables.
\subsection{Results}
related to the research questions and/or hypothesis.\\
They should include: \\
-Percentages 
\\ -Statistical significance
\\ -P value should be mentioned P<=0.05. The less is more significant.
\\ -If there was a hypothesis, you should mention whether it was accepted or rejected.
\\ -If there are many hypothesis discuss each one.
\subsection{Practical Implications}
Real life applications mentioned in thee discussion/conclusion section. In case they are missing, state their absence. Can be found in the Discussions/Conclusions section.
\subsection{Evaluation}
Discuss one weakness and one strength points.\\
An element was missing/not mentioned/ not written well. \\ 
and Their justification as well.


\end{document}
