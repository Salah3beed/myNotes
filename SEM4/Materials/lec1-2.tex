\documentclass[13]{article}
\newtheorem{exer}{Question}
\usepackage[margin=0.5in]{geometry}
\newtheorem{sln}{Solution}
\usepackage{graphicx}
\newtheorem{note}{Note}
\begin{document}
\section{Lecture 1}
\begin{exer}
Material Science vs Materials Engineering ?
\end{exer}
\textbf{Materials Science} involves investigating relationships between structures and properties of materials. While \textbf{Materials Engineering} involves designing/engineering the structure of a material to produce a predetermined set of properties using relations of structure-body. So a material scientist   develops/synthesizes new materials whereas the engineer creates products using those existing materials and/or develop techniques for processing materials.
\begin{exer}
What is the structure of a material.
\end{exer}
\textbf{Structure} in general refers to the arrangement of material's internal components . \\
\textbf{Subatomic Structure} involves electrons within the individual atoms and other particles inside the nuclei.\\
\textbf{Atomic}: Organization of atoms or molecules relative  to one another. \\
\textbf{Microscopic}: Groups of atoms that are normally agglomerated together\\
\textbf{Macroscopic}: seen using the naked eye.
\begin{exer}
What is a Property of a material ?
\end{exer}
It's a material's trait in terms of the kind and magnitude of a response to a specific imposed stimulus. 
\begin{exer}
What are Mechanical Properties of a material ?
\end{exer}
It's the response or deformation of a material due to an applied load or force.
\begin{exer}
Mechanical Properties Standards ?
\end{exer}
Mechanical properties are measured using standardized testing
techniques coordinated by professional societies such as the ASTM American Society for Testing and Materials. 
\begin{exer}
Processing and Performance?
\end{exer}
\textbf{Processing} refers to the methodology or techniques by which a material is prepared. \\
\textbf{Performance} indicates how good the materials will do its function.
\begin{exer}
Processing and Performance and so on relations.
\end{exer}
The \textbf{structure}  of a material depends on how it is \textbf{processed} , and material's \textbf{performance} is a function of its \textbf{properties} . And the properties is affected by the structure. 
\begin{exer}
Classification of Materials
\end{exer}
C-MPC
\begin{itemize}

\item Metals
\begin{itemize}

	\item Dense atomic packing (Metallic bonding)
	\item Non localized electrons (good conductors)
	\item Metals are dense because they are made of heavy atoms, packed 
densely together (iron, for instance, has an atomic weight of 56).
\end{itemize}
\item Ceramics
\begin{itemize}

\item Compounds of metallic and non metallic elements
\item Brittle
\item Ceramics, for the most part, have lower densities than metals because
	they contain light nonmetals like O, N or C atoms.
\end{itemize}
\item Polymers
	\begin{itemize}
	
	\item Most are organic compounds based on C and H
	\item Low Density, and Ductile.
	\item Polymers have low densities because they are largely made of
		light carbon (atomic weight: 12) and hydrogen (atomic weight:
		1) 	
	\end{itemize}
\item Composites
\begin{itemize}

\item Physical binding between materials from other classes to get intermediate properties.

\end{itemize}
\end{itemize}
\begin{exer}
What does Failure mean ?
\end{exer}
The end goal of studying property structure and processing relationships is that a product or component can perform its function and not fail in service. \\ Failure does not necessarily mean fracture. Failure to perform a function can be due to
\begin{itemize}

	\item \textbf{Excessive Elastic deformation:}  excessive deflection of closely mating
	parts can result  in interference and damage to the parts, this type
	of failure is controlled by the modulus of elasticity, not by the
	strength of the material. Generally, little metallurgical control can
	be exercised over the  elastic modulus. The most effective way to
	increase the stiffness  of a member is usually by changing its shape
	and increasing the dimensions of its cross section.
\item \textbf{Yielding and excessive plastic deformation:}  occurs when the elastic
	limit of the metal has been  exceeded. Yielding produces permanent
	change of shape, which may prevent the part from 
\item \textbf{Fracture:}  The formation of a crack which can result in complete
	disruption of continuity of the member  constitutes fracture. A part
	made from a ductile metal which is loaded statically rarely
	fractures because it will first fail by excessive plastic
	deformation.  Fracture could be a combined effect of stress and other
	factors such as corrosion for example.


\end{itemize}
\end{document}
